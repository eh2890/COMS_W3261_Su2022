\documentclass{../homework}

\title{COMS 3261 Homework 0}
% \author{Tim Randolph}
\date{Summer A 2022}

\begin{document}

\maketitle

\textbf{This problem set is OPTIONAL. } The purpose is to test yourself and review discrete mathematics concepts that will be essential for the course. \\

% This homework is due \textbf{[date]}. Submit to GradeScope (course code: 2KGDW8).

\textbf{Grading policy reminder:} \LaTeX~is preferred, but neatly typed or handwritten solutions are acceptable.\footnote{The website \href{https://www.overleaf.com/}{Overleaf} (essentially Google Docs for LaTeX) may make compiling and organizing your .tex files easier. Here's a quick \href{https://www.overleaf.com/learn/latex/Learn_LaTeX_in_30_minutes}{tutorial}.} I recommend using the .tex file for the homework as a template to write up your answers. Your TAs may dock points for indecipherable writing.\\

Proofs should be complete; that is, include enough information that a reader can clearly tell that the argument is rigorous. \\

If a question is ambiguous, please state your assumptions. This way, we can give you credit for correct work. (Even better, post on Ed so that we can resolve the ambiguity.) \\

The tool \url{http://madebyevan.com/fsm/} may be useful for drawing finite state machines. \href{https://detexify.kirelabs.org/classify.html}{Detexify} is a nice tool that lets you draw a symbol and returns the \LaTeX~codes for similar symbols. The website \href{https://www.mathcha.io/}{mathcha.io} allows you to draw diagrams and convert them to \LaTeX~code. (You'll need to add the command ``\textbackslash usepackage\{tikz\}'' to the preamble of your .tex file to import the package tikz.)



\clearpage
\section{
    Problem 1 (0 points).
}
    

\begin{enumerate}
    \item Examine the following formal descriptions of sets so that you understand which members they contain. Write a short informal English description of each set.
        \begin{enumerate}
            \item $\{1, 3, 5, 7...\}$.
            
            \item $\{n \; | \; n = 2m$ for some $m \in \mathbb{N}\}$. (Here $\mathbb{N}$ denotes the natural numbers $\{1, 2, \dots\}$.) 
            
            \item $\{w \; | \; w$ is a \emph{binary string} (a string of 0's and 1's) and $w$ equals the reverse of $w\}$.

        \end{enumerate}
    \item Write formal descriptions of the following sets.
        \begin{enumerate}
            \item The set containing the numbers 1, 10, and 100.
            
            \item The set containing all natural numbers less than 5.
            
            \item The set containing nothing at all. (There are two ways to write this: the ``natural'' way and using a certain special symbol.)
            
        \end{enumerate}
    \item If the set $A$ has $|A|$ elements and the set $B$ has $|B|$ elements, how many elements are in $A \times B$? (Here $\times$ denotes the \emph{Cartesian product}: $A \times B$ represents the set of all ordered pairs $(a, b)$ consisting of one element $a \in A$ and one element $b \in B$.)

    \item If $C$ is a set with $|C|$ elements, how many elements are in the power set $\mathcal{P}(C)$? (The power set $\mathcal{P}(C)$ is the set containing all subsets of $C$.)

    \item Let $A = \{1, 2, 3, 4, 5\}$ and $B = \{1, 3, 5, 7, 9\}$. What is $A \cap B$? $A \cup B?$ $A \setminus B$ (also written $A - B$)?\footnote{You may have seen $A \setminus B$ defined only when $B$ is a subset of $A$ ($B \subseteq A$.) In this course, we'll use $A \setminus B$ in a slightly broader sense to mean ``all the elements of $A$, minus any elements in $A$ that are also in $B$.''} What is the complement $\overline{A}$ of the set $A$, considered with respect to the universe of integers $\mathbb{Z}$?

    \item We use parentheses $()$ to distinguish $\emph{sequences}$, which care about the order of their elements, from sets. Of course, we can also make sets of sequences. What is $\{(a, b), \{a, b\}\} \cup \{(b, a), \{b, a\}\}$?
    
    \item Prove that
    \[
        A \cap ((B \cup \overline{A}) \cap \overline{B} ) = \emptyset.
    \]

\end{enumerate}

\blfootnote{ Rationale: The goal of this question is to make sure you're comfortable with the idea of sets, set notation and set operations. }
\blfootnote{ References: Sipser 0.2 pp. 3-6; \href{https://simple.wikipedia.org/wiki/Set}{Simple English Wiki: Set}.}


\clearpage
\section{Problem 2 (0 points).}

\begin{enumerate}
    \item Consider the undirected graph $G = (V, E)$, where $V$, the set of nodes, is $\{1, 2, 3, 4\}$ and $E$, the set of edges, is $\{\{1, 2\}, \{2, 3\}, \{1, 3\}, \{2, 4\}, \{1, 4\}\}$. Draw $G$. What are the degrees of each node? Indicate a path from node 3 to node 4 on the graph.
    
    \item How many components (connected pieces) does the graph 
    \[
        G = (\{1, 2, 3, 4, 5\}, \{ \{1,2\}, \{2,3\}, \{3,4\}, \{4,1\}, \{1,3\}, \{2,4\}\})
    \]
    contain? How many triangles does it contain?
    
    \item Prove that every \emph{tree} (connected graph without a cycle) that has $n \geq 1$ vertices has exactly $n-1$ edges.

\end{enumerate}

\blfootnote{ Rationale: The goal of this question is to make sure you're comfortable with graphs and proofs that refer to generic graphs. }
\blfootnote{ References: Sipser 0.2 pp. 10-13; \href{https://en.wikipedia.org/wiki/Graph_(discrete_mathematics)}{Wiki: Graph}.}




\clearpage
\section{Problem 3 (0 points).}

\begin{enumerate}
    \item Given $f:\mathbb{N} \cup \{0\} \rightarrow \mathbb{N} \cup \{0\}$ where $f$ is defined as: \\
    \[
        f(x)=
        \begin{cases}
        x+1 & \text{if } x \text{ is even} \\
        x-1 & \text{if } x \text{ is odd} \\
        \end{cases}
    \]
    Prove or provide a counterexample that $f$ is 
    \begin{enumerate}
        \item one-to-one (injective),
        \item onto (surjective),
        \item and bijective (both one-to-one and onto).
        
    \end{enumerate}
\end{enumerate}

\blfootnote{ Rationale: The goal of this question is to make sure you're comfortable thinking about discrete functions. }
\blfootnote{ References: Sipser 0.2 pp. 7-10, \href{https://en.wikipedia.org/wiki/Injective_function}{Wiki: One-to-one and Onto}, \href{https://en.wikipedia.org/wiki/Relation_(mathematics)}{Wiki: Relations}.}







\clearpage
\section{Problem 4 (0 points). }

Prove using \emph{contradiction} that for all integers $n$, if 5 divides $n^2$ then 5 divides $n$ [Hint: what does it mean to not be divisible by 5?]. 

\section{Problem 5 (0 points).}

Prove that for any positive integer $n$, there exists a sequence of n consecutive positive composite integers. [Hint: try to construct such a sequence!]

\blfootnote{ Rationale: The goal of this question is to make sure you're comfortable with proof by contradiction and construction. }
\blfootnote{ References: Sipser 0.4, pp. 21-22. See also Sipser 0.3 on strategies for finding proofs, and this \href{https://medium.com/@nissim.lavy/types-of-proofs-c43ffacc8ada}{Medium post} that introduces proof by construction, contradiction and induction with examples. }


\clearpage
\section{Problem 6 (Proof by Induction).}
\begin{enumerate}
    \item Let $S(n) = 1 + 2 + \cdots + n$ be the sum of the first $n$ natural numbers and let $C(n) = 1^3 + 2^3 + \cdots + n^3$ be the sum of the first $n$ cubes. Prove the following equalities by induction on $n$ to arrive at the curious conclusion that $C(n) = S(n)^2$ for every $n$.
    \begin{enumerate}
        \item $S(n) = \frac{1}{2} n (n+1)$.

        \item $C(n) = \frac{1}{4}(n^4 + 2n^3 + n^2) = \frac{1}{4}n^2(n+1)^2$.

    \end{enumerate} 
    \item Assume $n$ is a positive integer. Use induction to prove the following for all natural numbers $n$: 
    \[
        \frac{1}{1\cdot2} +  \frac{1}{2\cdot3} + \cdots +  \frac{1}{n\cdot(n+1)} = 1 - \frac{1}{n+1}.
    \]

\end{enumerate}

\blfootnote{ Rationale: The goal of this question is to make sure you're comfortable with proof by induction. }
\blfootnote{ References: Sipser 0.4, pp. 22-25. See also Sipser 0.3 on strategies for finding proofs, and this \href{https://medium.com/@nissim.lavy/types-of-proofs-c43ffacc8ada}{Medium post} that introduces proof by construction, contradiction and induction with examples. }

\end{document}

